%% content.tex
%%

%% ==============
Add additional content chapters if required. \chapter{Model Driven Software Development}
\label{ch:Content1}
%% ==============

The content chapters of your thesis should of course be renamed. How many chapters you need to write depends on your thesis and cannot be said in general. 

Check our the examples theses in the Wiki. 

Of course, you can split this .tex file into several files if you prefer. 


%% ===========================
\section{Common MDSD Concept and Terminology}
\label{ch:Content1:sec:Section1}
%% ===========================

What is MDSD
Book 
Terminology
How detailed an introduction to MDSD area?
Motivation
	Modeling4.1Common MDSD COncept and Terminology
		4.5 Software Factories
			4.8 Domain SPecific Modeling
			5.4 MDSD and Domain-Driven Design
			5.6 MDSD and Agile Software Development
			13.5 Product-Line Engineering
			
			Tools EMF, Eclipse
			9. Code generation ?

\dots

%% ===========================
\subsection{\ldots}
\label{ch:Content1:sec:Section1:subsec:SubSection1}
%% ===========================

\dots


%% ===========================
\section{Domain Specific Modeling}
\label{ch:Content1:sec:Section2}
%% ===========================

\dots
%% ===========================
\subsection{\ldots}
\label{ch:Content1:sec:Section2:subsec:SubSection1}
%% ===========================

\dots

%% ===========================
\section{Product-Line Engineering}
\label{ch:Content1:sec:Section3}
%% ===========================

\dots

%% ===========================
\subsection{\ldots}
\label{ch:Content1:sec:Section2:subsec:SubSection1}
%% ===========================

\dots

%% ===========================
\section{Develpoment Methodologies}
\label{ch:Content1:sec:Section4}
%% ===========================

\dots

%% ===========================
\subsection{MDSD and Agile Software Development}
\label{ch:Content1:sec:Section4:subsec:SubSection1}
%% ===========================

\dots


%% content.tex
%%

%% ==============
\chapter{End User Installation of Sensors/Smart Environmets}
\label{ch:Content2}
%% ==============

\huge{} ITS ABOUT HOW TO GUIDE THE USER THROUGH THE INSTALLATION
\normalsize{}

Smart home systems are on the rise

Ubiquitous computing systems rely upon information obtained from sensors depoyed
in the environment to \hl{function/work}. In the past years these sensors have
become cheaper, smaller and \hl{increasingly available to end users}. With this
in mind, the dominant factor of deploying smart home systems is the installation
barrier. Hiring a professional to do the installation is very expensive and the
flexibility to adapt the system to the ever changing needs of the user are
greately limited. Unfortunately, as noted in \cite{Beckmann2004}, the user has
difficulties 

This paper gives an overview ?of the? 
Motivation : Financial aspect

\dots


%% ===========================
\section{Section 1}
\label{ch:Content1:sec:Section1}
%% ===========================

\dots


%% ===========================
\section{Section 2}
\label{ch:Content1:sec:Section2}
%% ===========================

\dots

%% ==============
\chapter{Environmental Smartness With Second Sighted Glasses?}
\label{ch:Content6}
%% ==============

%% ===========================
\section{Takuro Yonezawa}
\label{ch:Content6:sec:Section1}
%% ===========================

\dots

%% ==============
\chapter{Usability of User Interfaces Generated with ah Model-Driven
Architecture Tool?}
\label{ch:Content7}
%% ==============

%% ===========================
\section{\ldots}
\label{ch:Content7:sec:Section1}
%% ===========================

\dots

%% ==============
\chapter{User comprehension performance for dataflow-based rules in smart
Environments}
\label{ch:Content8}
%% ==============

%% ===========================
\section{\ldots}
\label{ch:Content8:sec:Section1}
%% ===========================

\dots

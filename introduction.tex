% % ==============
\chapter{Motivation}
\label{ch:Motivation}
%% ==============

Fortschritte in der Technologie haben Sensoren kosteng�nstiger und
leichter zu benutzen gemacht. Diese Fortschritte reduzieren die
Technologische Barriere f�r die Installation von Ubicomp-Systemen. Laut
\cite{Beckmann2004} beschr�nkt sich der Installationsprozess auf die
korrekte Platzierung und der semantischen Assoziation zwischen den
Knoten der Anwendung und den Teil der physikalischen Welt den sie
beobachten.
Professionelle Installationen sind im Vergleich zu den Kosten der
Anschaffung sehr gro� und k�nnen die Ausbreitung solcher Systeme
negativ beeinflussen. Die Endnutzer-Installtion bringt mehrere
Vorteile. Sie ist billiger und kann einem Ubicomp-System helfen
schneller in der breiten Masse benutzt zu werden.
Des weiteren lernen Benutzer das System besser kennen und k�nnen so
mehr Kontrolle dar�ber haben und sie k�nnen das System an ihre
speziellen Bed�rfnisse anpassen.

Die Systeme die uns umgeben werden flexibler, enthalten mehr
Intelligenz und sind gr��tenteils rekonfigurierbar. Dadurch entsteht
eine hoche Komplexit�t bei der Installation, Konfiguration oder
Wartung. Um den Nutzern eine Hilfestellung zu leisten k�nnte man
individuell angepasste Handb�cher erstellen. Dieser Ansatz kann jedoch
nur effizient sein wenn es eine automatisierte L�sung gibt.

AR Systeme im Prototypstadium die sehr intuitiv sind.

Kann das aber auch mit einer Textuellen Benutzeranleitung
funktionieren:
Modellgetrieben Handb�cher f�r
intelligente, rekonfigurierbare Systeme erstellen um damit den Nutzern
einen besseren �berblick �ber das System zu verleihen

Was ist der Vorteil ihres L�sungsansatzes? (auch gg�.anderen Ans�tzen)
Aktionen (Action): Welche Schritte wollen Sie gehen? Der Vorteil w�re
eine automatisierte toolchain die �nderungen in dem System in dem
Handbuch reflektieren und somit immer Aktuellen bleiben und dem Nutzer
eine Sicht auf das System erlauben.
% % ==============
\chapter{Einleitung}
\label{ch:Motivation}
%% ==============

% 
% 1,5 Seiten Max.
% Beispiel bei Einleitung, worum geht es �berhaupt?
% Weniger text, konkrete Probleme geh�ren nicht dahin.
% Problemstatement
% Was wird in dieser Seminararbeit bearbeitet
% Fragestellung.

% =============================================% Problem : \ldots in der
% Seminararbeit gucken wir uns an .. weil .. und
% =============================================%
Heutzutage erlaubt der technologische Fortschritt eine Vielzahl von Ger�ten mit
unterschiedlichen Anwendungs-, Vernetzungs- und Interaktionsm�glichkeiten
kosteng�nstig in der Heimautomatisierung einzusetzen. Als Anwendungsgebiet der
Smart-Envi\-ron\-ments zielt die Heimautomatisierung auf die Komfortsteigerung
der Nutzer, durch die Ausstattung des Eigenheims mit intelligenten, vernetzten
Ger�ten (Gadgets genannt). Gadgets sind flexibel, intelligent und gr��tenteils
rekonfigurierbar.
Aufgrund, der im System existierenden Interkationen zwischen Gadgets und der
gro�en Anzahl an Konfigurationsm�glichkeiten, entstehen komplexe
Interaktionsmuster, die eine hohe kognitive Last f�r den Menschen erzeugen. Die
Installation eines Heimautomatisierungssystems �berfordert daher den Anwender.
% =============================================%
Installationsanleitungen sollen dabei eine Abhilfe schaffen, wobei laut
\cite{Beckmann2004} das konzeptuelle Modell, dass sich in den Gedanken der
Nutzer w�hrend der Installation bildet, eine wichtige Rolle im Verst�ndnis
�bernimmt.
Wie in \cite{Antifakos2002} aber angedeutet, sind heutige Benutzerhandb�cher
nicht in der Lage auf die verschiedenen Bed�rfnisse oder auf die schon
vorhandene Erfahrung der Anwender und den aktuellen Kontext einzugehen.\\
% =============================================%

Die vorliegende Arbeit stellt im ersten Teil Erkentnisse aus Studien zu
Installationsaufgaben vor, um Anforderungen an installationsunterst�tzende
Systeme in Smart-Environments zu sammeln.
Im zweiten Teil der Arbeit sind Ans�tze und Werkzeuge pr�sentiert, die zur
flexiblen Erstellung von Dokumentationen und Anleitungen geeignet sind.
Anschlie�end findet eine Gegen�berstellung der Ans�tze und Werkzeuge, um ihre
Eignung bez�glich der Anforderungen festzustellen. Die abgeleiteten Erkenntnisse
sollen der Verbesserung von installationsunterst�tzenden Methoden und
Technologien dienen.

% Eine so erstellte Anleitung soll die Installation von Smart-Environments wie
% die Heimautomatisierung erleichtern, damit sie auch von gew�hnlichen Benutzern
% bew�ltigt werden kann.

% Erhebung von Anforderungen

% =============================================%
% Die in dieser Arbeit identifizierten Prinzipien zur Nutzerunterst�tzung bei
% Installationsaufgaben und Generierung von Dokumenten, sollen genutzt werden, um
% die Generierung einer textuellen Anleitung aus systembeschreibenden Modellen zu
% definieren. 
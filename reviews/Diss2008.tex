%% ==================================================
%% 2008 Diss Auto Generierung von Arbeitsablaeufen fur den Service an Produktionssystemen
%% Zice despre problema care o sa o 'rezolv' eu in SWC deci atentie!
% Formulare si idei pot fi preluate
*Generierung von Arbeitsanweisungen aus Modellen*

Die Arbeit beinhaltet keine Studie oder Industriestudie?

Es wird auf eine Dom�ne konzentriert und n�hmlich unterst�tzung von
Servicetechnikern bei Montagearbeiten.

\begin{description}
  \item[Dokumentation] Die Strukturierte Weitergabe von Produkt- und
  Prozesswissen
\end{description}
 
 
Aus Agarwala: Die �berpr�fung gegenseitiger Beinflussung von
Bauteilen (interference) und die M�glichkeit, ein Bauteil mit einem
anderen zu verbinden (Attachement)

% Agarwal Abstract
% Visions of smart homes have long caught the attention of
% researchers and considerable effort has been put toward enabling home
% automation. However, these technologies have not been widely adopted despite
% being available for over three decades. To gain insight into this state of
% affairs, we conducted semi-structured home visits to 14 households with home
% automation. The long term experience, both positive and negative, of the
% households we interviewed illustrates four barriers that need to be addressed
% before home automation becomes amenable to broader adoption. These barriers
% are high cost of ownership, inflexibility, poor manageability, and difficulty
% achieving security. Our findings also provide several directions for further
% research, which include eliminating the need for structural changes for
% installing home automation, providing users with simple security primitives
% that they can confidently configure, and enabling composition of home devices.
% Author

Taskflows sind Arbeitsabl�ufe und erm�glichen ein schrittweises
Erf�llen der erforderlichen Aufgaben.
In der Arbeit von \cite{} werden Taskflows in Fragmente unterteilt.
Ein Fragment stellt eine logische Einheit von Szenen bzw. T�tigkeiten
dar. Fragmente l�sen nach ihrer Durchf�hrung eine Zustands�nderung
aus, i.e, sie bringen das System aus einen definierten Zustand in den
n�chsten.
Fragmente sind wiederrum in Szenen unterteilt, die Unterst�tzung f�r
eine einzelne T�tigkeit darstellen. Sie enthalten die eigentlichen
Anweisungen f�r den Servicetechniker.

Taskflow - Aufgabenebene
Fragmenete - Zustandsebene
Szenen - T�tigkeitsebene

Deklaratives Wissen vs Prozedurales Wissen
Begriffe, Objekte, Relationen, Constraints, Regeln VS
handlungsleitend, Algorithmen

Meta-Daten Ontologie Duplin-Core DCMI 2000

'Ein Reasoner die die definierten Bedingungen interpretiert und die
Klassenhierarchie normalisiert' pg 61

Inca nu prea am inteles cum naiba face generarea asta: prima data
cauta fragmente goale si apoi le umple ? 

Er baut Regeln die er dann mit Hilfe der Ontologie pr�ft/enforced 

mereotopologisch - mereologisch und topologisch
mereologisch - behandelt Teile und die Ganzen die daraus gebildet werden

Formalisierung von mereotopologischen Beziehungen mit Hilfe von
Predikaten (S. 69) - kann n�tzlich sein denn diese Beziehungen gibt es
auch in Software-Komponenten

Ontologie aus Pr�dikaten aufgebaut

Benutzt OWL

Modelliert die Ontologie und generiert regeln die spezifisch f�r
Zusammengesetzte Bauteile sind. Es ist m�glich diesen Ansatz zu
generalisieren in dem man ein System als ein Zusammengesetztes ganzes

Was ist nochmal genau die Anreichrung?

Positions vs Beziehungsabh�ngigkeit
sieht?
 

 Benutzt pr�dikatenlogik. Benutzt Planungsalgorithmen. In Kapitel 5.1
 werden Randbedingungen aufgef�hrt


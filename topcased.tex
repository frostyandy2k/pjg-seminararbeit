
%% ==============
\section{Modellgetriebene Generierung von Dokumentation}
\label{ch:sec:}
%% ==============

Was ist topcased?
Wozu wird es benutzt/Wozu wurde es gebaut?

\cite{Pontisso06}

Modellierung standardisiert OMG MDA, graduelle verfeinerung
Anstrengungen der MDA si

MDSD an diesem Beispiel kurz erkl�ren

[Einleitung] Ein weiteres Werzkeug zur automatischen Generierung von
Dokumentation ist das in der Topcased Plattform enthaltene Gendoc, das aus
Quellmodellen modellgetriebene Erstellung von Dokumentation erm�glicht.

Topcased\footnote{http://www.topcased.org/ ab 2014 auf http://polarsys.org/} ist
ein Open-Source Werkzeug f�r modellgetriebene Entwicklung und dient dem System-
und Software-Engineering von kritischen und embedded Anwendungen. Die Plattform
unterst�tzt alle Entwicklungsstufen von der Anforderungsanalyse bis zu der
Implementierung. Sie basiert auf der Eclipse Plattform und benutzt das Eclipse
Modeling Framework (EMF), um einen modellgetriebenen Arbeitsfluss zu
realisieren. Es unterst�tzt neben UML Modellen auch andere von der OMG
standardisierten Modellierungssprachen wie SysML\footnote{http://www.sysml.org/} und
ReqIF\footnote{http://www.omg.org/spec/ReqIF/}.

Als Teil der Topcased Plattform erlaubt Gendoc die Generierung von textueller
Dokumentation aus UML Modellen. Die Generierung wird von Skripten in den
Dokumenten-Templates, die Logik und Modellabfragen beinhalten, gesteuert. Um die
Modellabfrage zu realisieren sind Acceleo Skripte und OCL Ausdr�cke m�glich.

Acceleo\footnote{} Skripte, die direkt in den Dokumenten-Templates geschrieben
werden lesen die Quelldomodelle mit Hilfe von OCL aus und f�llen das Dokument.
Die benutzten Skripte, die in der Acceleo Sprache geschrieben sind, k�nnen
beliebig Komplexe OCL Ausdr�cke wie Schleifen oder Logik enthalten. 

Topcased unterst�tzt einen Gro�teil der UML Modelle, namentlich sind das
Klassen-, Komponenten- und Deployment-Diagrammes, die die Struktur beschreiben
und Sequenz-, Aktivit�ts- und Zustandsdiagramme die die dynamischen Teile eines
Systems beschreiben. Auch Use-Case Diagramme k�nnen modelliert, abgefragen und
in die Dokumentation mit eingebunden werden.

Der generative Ansatz verl�uft wie in Abbildung \ref{fig:topcased-gendocflow}
dargestellt. Die Modelle und die Dokumenten-Templates sind die Eingaben des
Systems, als Ausgabe liefert Gendoc das ausgef�llte Dokumenten-Template. Die
Abspeicherung der ODT\footnote{} und DOCX\footnote{} Formate als XML, erlaubt es
Gendoc die Skripte zu extrahieren und Auszuwerten. Da die Skripte in der
Modelltransformationssprache Acceleo geschrieben sind k�nnen direkt auf die
Modelle ausgef�hrt werden. Dieser Ansatz erlaubt es, alle von der EMF
unterst�tzten Modellformate in Gendoc einzusetzen.

UML Modelle liefern Informationen �ber den erw�nschten Einsatz des Systems und
werden prim�r in Zusammenarbeit mit Dom�nenrepr�sentaten erstellt.

Die Skripte sind in einer Markup-�hnlichen Sprache eingebettet, die durch  und
HTML �hnliche Unterst�tzung von Bildern und Listen besitzen. 

Beim Auswerten der Skripte behalten die Ausgaben die Formattierung bei.
Graphische Elemente, wie konkrete Diagramme k�nnen aus dem
Quellmodell generiert werden und neben dem Text in das generierte Dokument
eingef�gt werden.
Da die Formattierung des Dokuments beibehalten wird ist die dynamische
Verlinkung mithilfe der Dokumenteneigenen Mechanismen m�glich.

Gendoc Befehlen werden durch das <gendoc> tag gekennzeichnet. Neben dem Acceleo
Skript sind auch statische Texte mit Formatierungen, Bilder und Tabellen
Erlaubt.

Hyperlinking ist m�glich durch das Benutzen des Dokumentformat abh�ngigen
Verlinkungsmechanismus.

Die Konfiguration, die die Erstellung des Dokumentes steuert wird auch im
Dokument-Template unter einem Markup-Tag gespeichert.

Ein Mechanismus zur Wiederverwendung von Code in den Templates steht zur
Verf�gung.

Zugriff auf alle Elemente der Diagramme die benutzt weredn um das System zu
modellieren. Volle Kontrolle auf die Erstellung des Dokumentes und der
Informationen die aus den Modellen abgeleitet werden.

Kann es andere Fragmente mit einbeziehen? Sind Acceleo Skript und Dokument immer
gemischt. Dom�nenexperten k�nnen dann wahrscheinlich nicht damit arbeiten.

Sagen dass Topcased auch code generieren Kann

Gendoc erlaubt es auch im Batch-Modus zu laufen, um automatisierte
Werkzeugketten zu realisieren.

Nachteil: Code aus Templates kann nur mit gro�em Aufwand auf ein Anderes
Template angewandt werden.

\begin{figure}[htp]
\begin{center}
  \includegraphics[width=0.6\textwidth]{img/topcased-gendocflow.png}
  \caption[XXX]{XXXXXXX}
  \label{fig:topcased-gendocflow}
\end{center}
\end{figure}


Wie funktioniert es?
	Konzeptuell
	Technische realisierung

	
% Listing erkl�ren was die einzelnen tags machen. Are rost listingu? Vezi daca se
% merita la dopler.
% % =======
% \lstset{breaklines=true,language=XML,caption={Das Element
% \emph{doplerdocplaceholder} mit dem \emph{doplerdoc} Attribut dient als
% Platzhalter f�r den K�hlungsmechanismus \emph{cooling\_mech}
% \cite{Rabiser2010}},label=lst:dopler_docbook_xml}
% \lstinputlisting[language=XML]{listings/topcased.xml}
% % =======



%% ==============
\chapter{Ausblick}
\label{ch:NextSteps}
%% ==============

Usability studies machen
Am anfang um eine Hypothese zu erstellen und sehen was leute von so einem System
erwarten

 !System definieren!
 
Punkte finden die interessant sind/interessant zu beschreiben sind.
Formative studie - ?
Qualitativ, qunatitative?

Interviews erstellen -> Requirements sammeln -> Hypothese

Studien erstellen um zu sehen ob eine statistisch signifikanter Unterschied
zwischen den statisch erstellten und den dynamisch generierten besteht.

Wizard of Oz studie => Referenzimplementierung machen

Idee:
	Seite erstellen die einen konfigurator darstellt. User konfigurieren lassen.
	Nach einiger zeit den usern das ding in die Hand dr�cken und arbeten lassen.
	
2x5/7 Leute sollte man haben f�r eine Usability studie
Gruppen Interviews sind nicht schlecht um Gruppendynamik zu f�rdern.
Tutorium Users study?

Methodik muss fest sitzen!

Eriksson Semantic document approach
Knuth Literate Programming
